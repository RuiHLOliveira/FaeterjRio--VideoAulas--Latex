%%%%%%%%%%%%%%%%%%%%%%%%%%%%%%%%%%%%
%%%    Preâmbulo do Documento    %%%
%%%%%%%%%%%%%%%%%%%%%%%%%%%%%%%%%%%%

\documentclass[12pt]{article}
\usepackage[utf8]{inputenc} %Acentuação das palavras
\usepackage{helvet} %importando fonte helvet/arial equivalente
\renewcommand{\familydefault}{\sfdefault} %aplicando fonte helvet
\usepackage{geometry}
\geometry{
	a4paper,
	left=3cm,
	top=3cm,
	right=2cm,
	bottom=2cm
}
\usepackage{graphicx}
\renewcommand{\figurename}{Figura}
\renewcommand{\listfigurename}{Lista de Figuras}

%   ***1


\title{Modelo\\Especificação dos Requisitos do Software\\ Sistema de Teste LaTeX}
\author{\\ \\ \\Rui Henrique Leite de Oliveira \\ FAETERJ\\ \\ \\ \\}
\date{Rio de Janeiro - RJ \\ 3 de Março de 2016}

\begin{document}

%   ***3

\maketitle
\newpage

%   ***4, 5


\listoffigures
\newpage

%   ***2, 5




\section{Introdução}
	\subsection{Objetivos deste documento}
		\paragraph{}Descrever e especificar as necessidades levantadas pelo autor que devem ser atendidas pelo produto Sistema de Teste LaTeX, bem como definir para os desenvolvedores o produto a ser feito.
		
		%   ***5
		\begin{figure}
			\caption{UML Logo}
			\centering
			\includegraphics[width=5cm]{"uml".png}
		\end{figure}
		
		\paragraph{}Público-alvo: cliente, usuários e desenvolvedores do projeto Taskando.
	\subsection{Escopo do produto}
		\subsubsection{Nome do produto e de seus componentes principais}
			\paragraph{}Sistema de Teste LaTeX (Componente único).
		\subsubsection{Missão do produto}
			\paragraph{}Apoio informatizado na pesquisa à ser realizada, fornecendo relatórios dos resultados obtidos.
			
			%   ***6
			\begin{figure}
				\caption{HTML5 Logo}
				\centering
				\includegraphics[width=3cm]{"imagens/html5".png}
			\end{figure}
				
		\subsubsection{Limites do produto}
			\paragraph{}O Sistema de Teste LaTeX terá interações somente via terminal ou interface gráfica desktop.
			\paragraph{}O Sistema de Teste LaTeX não terá suporte à funcionalidades online.
			\paragraph{}O Sistema de Teste LaTeX não persistirá informações utilizando SGBDs.
			\paragraph{}O Sistema de Teste LaTeX não terá ajuda on-line.
\end{document} 